% This section describes all procedures regarding the required data. Section \ref{holdings} describes the construction of the mutual fund holdings database and corresponding screening procedures. Section \ref{stock} describes the linkages of stocks to the holdings database. Section \ref{flows} describes the construction of fund flows and an analysis on fund flow predictability. 
\label{holdings}
 This section describes our data set and sample selection criteria, followed by summary statistics for the mutual fund sample including the firm characteristics obtained from the holdings of mutual funds. As a preliminary indication of the importance of characteristics, we run cross-sectional regressions of individual mutual fund returns on their factor betas and characteristics using the traditional two-pass procedure of \citet{fama1973risk} with rolling estimation of factor betas. To address the inherent errors-in-variables (EIV) bias, we use the EIV-corrected estimator of \citet{chordia2015cross}. 

\subsection{Data Selection}
The mutual fund database is constructed by combining the Center for Research in Security Prices (CRSP) Survivor-Bias-Free U.S. Mutual Fund database with the Thomson Reuters Mutual Fund Holdings S12 database, formerly known as CDA/Spectrum. To combine these databases, we rely on the MFLINKS database provided by Russ Wermers on Wharton Research Data Services (WRDS). The main focus is on U.S. equity actively managed mutual funds, for which the data on holdings is the most complete and reliable; we eliminate balanced, bond, money market, international, index, and sector funds, as well as funds not invested primarily in common stocks (for details on our selection, see Appendix A). 
\par 
We obtain both monthly and daily mutual fund returns from the CRSP mutual fund database. Additional stock-level information on the fund holdings are retrieved from the CRSP monthly stock file and the Compustat database. Consistent with previous literature, this study only considers common stocks traded on the New York Stock Exchange (NYSE), American Stock Exchange and NASDAQ; we excluding real estate trusts, foreign companies, closed-end funds and primes (we only retain shares codes 10 or 11). To mitigate the influence of return outliers we exclude stocks with prices below \$5 as of the fund holdings report date. The final database for the empirical analysis consists of monthly panel data on mutual fund holdings at the intersection of CRSP, Compustat and Thomson Reuters, spanning the period from January 2001 to December 2016. 

% \begin{singlespacing}

% \begin{table}[h!]
% \centering
% \small
% {\captionsetup{justification=centering,singlelinecheck=off}
% \caption{\bfseries Summary statistics of equity mutual fund sample} }
% \caption*{This table presents the summary statistics for the equity mutual funds sample over the period January 2001 to December 2016. Panel A reports statistics on the sample size. Panel B reports additional information on the funds. Panel C reports the cross-sectional distribution of characteristics averaged across all sample months. We weight each firm characteristic according to its current portfolio weight and calculate a fund's portfolio-weighted average characteristic. Market capitalization (Mcap) is the product between the previous month-end stock price and the previous month-end total shares outstanding. Book-to-Market (B/M) is the ratio between the most recently available book value of equity and the previous month-end market capitalization. Momentum (Mom12) is the past twelve-month cumulative return excluding the most recent month. Operating profitability (Profit) is the current revenues minus costs of goods sold, interest expense, selling, general, and administrative expenses, divided by book equity for the last fiscal year $t-1$.  Asset growth (Invest) is the percentage change in total assets from fiscal year $t-2$ to fiscal year $t-1$. Panel D reports the time series averages of the cross-sectional distributions of six-factor betas, which are estimated from rolling time series regressions using the past two years of daily fund returns. Panel E reports the time series averages of the cross-sectional correlations between factor betas and characteristics.}
% \label{my-label}
% \begin{tabular}{lrrrrr}
% \hline
%                                                 & Mean     & \begin{tabular}[c]{@{}l@{}}25\%\\ percentile\end{tabular} & Median   & \begin{tabular}[c]{@{}l@{}}75\%\\ percentile\end{tabular} & \begin{tabular}[c]{@{}l@{}}Standard \\ deviation\end{tabular} \\ \hline
% \multicolumn{6}{l}{Panel A: Observations}                                                                                                                               \\
% Number of distinct funds                        & 2,871    &                                                         &          &                                                           &                                                               \\
% Number of fund-report dates                     & 92,903  &                                                         &          &                                                           &                                                               \\
% Number of fund-month dates                      & 314,362   &                                                         &          &                                                           &                                                               \\
% Number of distinct stocks                       & 7,952    &                                                         &          &                                                           &                                                               \\
%                                                 &          &                                                         &          &                                                           &                                                               \\
% \multicolumn{6}{l}{Panel B: Fund characteristics}                                                                                                                                                                                                           \\
% Fund age                             & 22.90   & 15.76 & 19.96  & 25.38  & 12.33   \\
% Fund monthly net return (in \%)      & 0.51    & -2.08 & 0.99   & 3.60   & 5.12    \\
% TNA (total net assets) (in millions) & 1,341.43 & 56.00 & 220.00 & 863.10 & 5,335.74 \\
% Expense ratio (in \%)                & 1.24    & 0.97  & 1.19   & 1.45   & 0.63    \\
% Turnover ratio (in \%)               & 83.19   & 34.00 & 62.00  & 105.00 & 101.68                                                \\
%                                                 &          &                                                         &          &                                                           &                          


%                                   \\
% \multicolumn{6}{l}{Panel C: Fund holdings characteristics}                                                                                                                                                                                           \\
% Mcap (in millions) & 44,475.74 & 3,792.98 & 43,016.63 & 78,966.25 & 39,731.13 \\
% B/M                 & 0.46     & 0.32    & 0.44     & 0.56     & 0.16     \\
% Mom12  (in \%)      & 17.12    & 9.15    & 14.89    & 22.86    & 12.13    \\
% Profit              & 0.46     & 0.35    & 0.42     & 0.50     & 0.24     \\
% Invest              & 0.09     & 0.07    & 0.09     & 0.12     & 0.04                                               \\

%                                                 &          &                                                         &          &      

%                                                     &                                                               \\
% \multicolumn{6}{l}{Panel D: Rolling window fund six-factor betas}                                                                                                                                                                                                           \\
% $\beta_{MKT}$ & 0.98  & 0.94  & 0.99  & 1.03 & 0.10 \\
% $\beta_{SMB}$ & 0.20  & -0.08 & 0.07  & 0.46 & 0.34 \\
% $\beta_{HML}$ & 0.00  & -0.16 & 0.01  & 0.16 & 0.21 \\
% $\beta_{WML}$ & 0.02  & -0.05 & 0.01  & 0.09 & 0.12 \\
% $\beta_{RMW}$ & -0.06 & -0.17 & -0.02 & 0.09 & 0.21 \\
% $\beta_{CMA}$ & -0.04 & -0.15 & -0.02 & 0.10 & 0.21 \\ \hline
% \end{tabular}
% \end{table}
% \end{singlespacing}

% The mutual fund database is constructed by combining the Center for Research in Security Prices (CRSP) Survivor-Bias-Free U.S. Mutual Fund database with the Thomson Reuters Mutual Fund Holdings S12 database, formerly known as CDA/Spectrum. The former database contains mutual fund characteristics and the latter database contains mutual fund holdings. To combine both databases we rely on the mapping provided in MFLINKS. Stock data on the mutual fund holdings is obtained from the CRSP monthly file, supplemented by accounting data from Compustat. The final database for the empirical analysis consists of monthly panel data on mutual fund holdings at the intersection of CRSP, Compustat and Thomson Reuters, spanning the period from April 1980 to December 2015. Prior literature using this database include the works of \citet{wermers2000mutual}, \citet{lou2012flow} and \citet{jiang2014information}. 
 
% \subsection{Mutual fund selection}
% \par The main focus is on domestic equity actively managed mutual funds, for which the data on holdings is the most complete and reliable.  These funds are identified based on Lipper Objective codes, Strategic Insight Objective codes and Wiesenberger Fund Type codes.\footnote{We select funds with one of the following objective codes available in the CRSP Mutual Fund Database. 
% Lipper Objective codes: EI, EIEI, EMN, FLX, G, GI, I, LCCE, LCGE, LCVE,
% LSE, MC, MCCE, MCGE, MCVE, MLCE, MLGE, MLVE, SCCE,
% SCGE, SCVE, SESE, SG. 
% Wiesenberger Fund Type codes: SCG, AGG, G, G-S, S-G, GRO, LTG, I, I-S, IEQ,
% ING, GCI, G-I, G-I-S, G-S-I, I-G, I-G-S, I-S-G, S-G-I, S-I-G,
% GRI, MCG. 
% Strategic Insight Objective codes: SCG, GRO, AGG, ING, GRI, GMC.} Since some funds misreport their objective code, we require funds to hold at least 80\% and at most 105\% in common stocks, on average. Index funds are eliminated based on the CRSP index fund flags (provided since 2003) and by screening fund names.\footnote{Funds are dropped if the fund name contains the following strings: INDEX, IND, INDX, IDX, IDX, MKT, MARKET, S&P, SP, MSCI, NYSE, RUSSELL, NASDAQ, ISHARES, DOWJONES, SPDR, ETF, 100, 400, 500, 600, 1000, 1500, 2000, 3000, 5000.} Following \citet{kacperczyk2008unobserved}, to address the incubation bias\footnote{\citet{evans2004does} and \citet{kacperczyk2008unobserved} detect a form of survival bias in the CRSP mutual fund database, which stems from fund families sugarcoating their past performance. Fund families incubate private funds and only report the returns of the surviving incubated funds and do not disclose the past performance of terminated funds.} we delete observations for which the date of the observation is prior to the reported fund-start date and we delete observations with missing fund names. Other data from the CRSP Mutual Fund database include the total net assets (TNA), monthly net returns, fees, and other qualitative fund data. We aggregate all different share classes belonging to a single mutual fund at each point in time into one observation. Regarding the quantitative attributes of funds, we sum the TNA and we take a weighted average of the fund returns, expense ratio, turnover ratio and fees, using the lagged TNAs of each individual share class as weights. Regarding the qualitative attributes of funds (e.g., fund name, CRSP objective code, year of origin), the data of the oldest fund is retained. 

% % The data on fund return is reported every month, while fund's TNA is reported on a quarterly/annual basis before 1992 and on a monthly basis after. Fund fees, such as expense ratio,  turnover ratio and 12(b)1 fees are reported annually before 1999 and quarterly after 1999.

% \par Mutual fund holdings are provided by Thompson Reuters and are compiled from mandatory SEC fillings\footnote{Investment companies, which include mutual funds, insurance companies, banks, pension funds, and numerous other institutions are often called 13f institutions. These institutions are required to fill in a form with the Securities and Exchange Commission (SEC) on a semiannual basis.} and voluntary disclosures. From this database we exclude funds with the following objective codes: International, Municipal Bonds, Bond \& Preferred, Balanced and Metals. Every fund files the SEC form at the end of a quarter (the file date), which is often in the same quarter as the report date; the date for which the holdings are actually held (adjusted for stock splits\footnote{Adjustment are made using the cumulative adjustment factor for shares in the CRSP monthly file.}). To create a monthly time series of fund holdings, we keep reported holdings constant between report dates (e.g., holdings reported at the end of September are valid in October, November and December). A majority of funds report holdings on a quarterly basis, while a small number of funds have gaps between report dates of more than two quarters. To fill these gaps (of no more than four quarters), we impute holdings of missing quarters using the most recently available report date, assuming that these funds adopt a buy-and-hold strategy. To be included in the sample, we require funds to have consecutive holdings data for at least 36 months. 

% % In the final database about 65\% of the funds disclose their holdings quarterly, 34\% semi-annually and 1\% on a less frequent basis.

% % A problem comes from late reporting or stale data, which reflects cases where the report and file date are not in the same quarter. There are cases of multiple file dates corresponding to the same report date (and the same holdings), which lead to gaps in holdings data. I delete all duplicate records of fund-report dates. 

% % To obtain monthly funds holdings, I assume reported holdings are held in the subsequent months (no more than 3 months) up until the next report date (e.g., holdings reported at the end of September are valid for October, November and December).
% \par To combine the CRSP Mutual Fund database with the Thomson Reuters database, we use the MFLINKS provided by Russ Wermers on Wharton Research Data Services (WRDS). MFLINKS maps CRSP fund identifiers to Thomson Reuters fund identifiers, covering approximately 98\% of the domestic equity mutual funds. We manage to link about 92\% of the target universe in the CRSP Mutual Fund database to holdings data from Thomson Reuters. To ensure a reliable linkage between the two databases, we require that the TNAs reported by both databases do not differ by more than a factor of two. Finally, funds with less than 10 identified stock positions and less than \$5 million assets under management are excluded.  

% % I manage to link 88.01\% of all fund-holdings observations to a CRSP stock by matching the CUSIP reported by Thomson Reuters with either the historic CUSIP (NCUSIP) or the header CUSIP (HCUSIP) reported in the CRSP monthly file. 
% \par The final mutual fund database contains 2,708 distinct mutual funds including 108,576 fund-report dates and 384,418 fund-month observations. Table 1 presents the number of funds at the end of each year along with the TNA and number of holdings reported by Thomson Reuters.  There is a rising trend in both the number of funds, the average fund size and the market share held. Panel B of Table 2 presents main fund attributes including TNA, return, age, expenses and turnover. 
% Fund size reported by CRSP can differ from the fund size reported by Thomson Reuters as the latter rarely contains entire holdings of a fund. To investigate the magnitude of missing holdings, we compute the  difference\footnote{The percentage difference between both TNAs is calculated as $ \frac{|TNA_{CRSP}-TNA_{TR}|}{0.5*(TNA_{CRSP}+TNA_{TR})} $ } between TNAs reported by both databases, which is equal to 3.75\%, on average. Thus, the mutual funds sample contains the vast majority of equity holdings. 
\subsection{Mutual Fund Holdings Characteristics}
 For each stock in a fund's portfolio, we obtain firm characteristics, including market capitalization, book-to-market ratio, momentum return, operating profitability and asset growth. Market capitalization (Mcap) is defined as the product between the previous month-end stock price and the previous month-end  total shares outstanding. Book-to-Market (B/M) is the ratio between the most recently available book value of equity and the previous month-end market capitalization.\footnote{We supplement the book values from Compustat with hand-collected data provided by Moody's. It includes the data used in \citet{davis2000characteristics} and contains data ranging from 1926 to 2001. The data is available on \ULurl{ http://mba.tuck.dartmouth.edu/pages/faculty/ken.french/data_library.html.}} Momentum (Mom12) is the past twelve-month cumulative return over the period from month $t-12$ to $t-2$, where the most recent month is excluded to avoid short-term reversal effects. Operating profitability (Profit) is the current revenues minus costs of goods sold, interest expense, selling, general, and administrative expenses, divided by book equity for the last fiscal year $t-1$. Asset growth (Invest) is the percentage change in total assets from fiscal year $t-2$ to fiscal year $t-1$. We assume that all the accounting variables, e.g., book value of equity, operating profitability and asset growth, are publicly available six months after the fiscal year-end. 
 
 For each fund in our sample, we use individual stock holdings to compute the monthly fund-level market capitalization, book-to-market ratio, momentum return, operating profitability, and asset growth. We weight each firm characteristic according to its current portfolio weight and calculate a fund's portfolio-weighted average characteristic. For each characteristic, values greater than the 0.995 percentile or less than the 0.005 percentile are set equal to the 0.995 and the 0.005 percentiles each month. 
 
Table \ref{table1} reports statistics on the characteristics and the factor betas. Each month, we calculate the cross-sectional mean, standard deviation and percentiles for each characteristic and factor beta. We report the time series averages of the monthly cross-sectional distribution for characteristics in Panel C and for factor betas in Panel D. The factor betas are estimated from a six-factor model which augments the \citet{carhart1997persistence} four-factor model with the profitability and investment factors from \citet{FAMA20151}.\footnote{The risk factor returns for all factor models are provided on Ken French's Website.} The factor betas are derived from rolling-window regressions using daily returns of the past two years.
 
 Panel E of Table \ref{table1} reports the cross-sectional correlations between the factor betas and the characteristics averaged across all sample months. Conform to expectations, the betas for SMB and CMA are negatively correlated with their underlying characteristics, while the betas for HML, WML, and RMW are positively correlated with their underlying characteristics. The correlations vary in magnitude, with the highest correlations for the size and value factors (-0.82 and 0.75), the momentum and investment factors yield slightly lower correlations (0.57 and -0.47), and a modest correlation of 0.14 for the profitability factor. These correlations indicate that the factor betas and the characteristics do not convey identical information on the expected return, especially for the profitability factor. As a consequence, regressing fund returns on the risk factors may not fully adjust performance for the main anomalies.
 
%  \begin{singlespacing}
%  \begin{table}[h!]
% \centering
%  {\captionsetup{justification=centering,singlelinecheck=off}
%  \caption*{\bfseries Table 1 (Continued) }}
%  \caption*{\centering \small{Panel E: Cross-correlations between six-factor betas and characteristics}}

%  \small
% \label{my-label}
% \begin{tabular}{lrrrrrrrrrr}

%       \hline
%       & \text{$\beta_{SMB}$}  &\text{$\beta_{HML}$}   &\text{$\beta_{WML}$}   & \text{$\beta_{RMW}$}  & \text{$\beta_{CMA}$}   & Mcap   & B/M    & Mom12   & Profit & Invest \\ \hline
% $\beta_{MKT}$   & 0.096 & -0.093 & 0.113  & -0.204 & -0.173 & -0.023 & -0.114 & 0.170  & -0.004 & 0.077  \\
% $\beta_{SMB}$   &       & 0.095  & 0.043  & -0.142 & -0.075 & -0.816 & 0.110  & 0.283  & -0.255 & -0.064 \\
% $\beta_{HML}$   &       &        & -0.315 & 0.610  & 0.328  & -0.042 & 0.746  & -0.236 & -0.079 & -0.452 \\
% $\beta_{WML}$   &       &        &        & -0.231 & -0.270 & -0.054 & -0.469 & 0.568  & 0.103  & 0.191  \\
% $\beta_{RMW}$   &       &        &        &        & 0.361  & 0.246  & 0.367  & -0.283 & 0.136  & -0.274 \\
% $\beta_{CMA}$   &       &        &        &        &        & 0.049  & 0.354  & -0.215 & -0.004 & -0.470 \\
% Mcap   &       &        &        &        &        &        & 0.049  & 0.354  & -0.215 & -0.004 \\
% B/M    &       &        &        &        &        &        &        & -0.363 & -0.189 & -0.486 \\
% Mom12   &       &        &        &        &        &        &        &        & -0.003 & 0.054  \\
% Profit &       &        &        &        &        &        &        &        &        & 0.049  \\ \hline
% \end{tabular}
% \end{table}
% \end{singlespacing}

\subsection{The Importance of Characteristics}
\label{section2C}
We examine the role of factor betas and firm characteristics in explaining the returns of mutual funds. For this purpose we run monthly cross-sectional regressions of individual mutual fund returns on their factor betas and characteristics following the traditional two-pass procedure of Fama \& MacBeth (1973) with rolling estimation of factor betas. The monthly regressions are of the form: 
\begin{equation}
\label{csr}
R_{t} = \gamma_{0t} + \gamma_{1t}\hat{B}_{t-1} + \gamma_{2t}Z_{t-1} + \xi_{t}, \hspace{0.2cm} t=1,...,T,
\end{equation}
where $R_t$ is a $N_t$ $\times$ 1 vector of excess fund returns, $\hat{B}_{t-1}$ is a $N_t$ $\times$ $K$ matrix of estimated factor betas from a $K$-factor asset pricing model using the past two years of daily ending with month $t-1$, and $N_t$ is the number of funds in month $t$. The $N_t$ $\times$ $L$ matrix $Z$ contains lagged characteristics including the logarithm of market capitalization (Mcap), the logarithm of book-to-market ratio (B/M), the logarithm of one plus the past twelve-month cumulative return (Mom12), operating profitability (Profit), and asset growth (Invest).
We employ the errors-in-variables (EIV)-corrected estimator of \citet{chordia2015cross} to obtain coefficient estimates of $\hat{\Gamma}_{t}$ = ($\hat{\gamma}_{0t}$, $\hat{\gamma}_{1t}$, $\hat{\gamma}_{2t}$) in Eq.(\ref{csr}). We describe the EIV-corrected estimator in Appendix B.
\par We present the results for the CAPM, the \citet{fama1993common} three-factor model, the \citet{carhart1997persistence} four-factor model, the \citet{FAMA20151} five-factor model and a six-factor model with all factors. Table \ref{table2} reports the estimated coefficients on both factor betas and characteristics with the EIV bias correction and without. We find that all factor premia except for the value premium are insignificant when their underlying characteristics are added to the model. Momentum return (Mom12) is significant in all factor models which exclude the momentum factor with t-statistics just above 2. \citet{lou2012flow} and \citet{vayanos2013institutional} argue that winning funds, by scaling up their existing holdings that are concentrated in past winning stocks, drive up the returns of past winning stocks and thereby enhance the subsequent return of past winning stocks. This self-enhancing mechanism may lead to higher momentum returns for funds. 

We use the measure of \citet{lindeman1980introduction}\footnote{The measure of \citet{lindeman1980introduction} uses sequential sums of squares from the linear model. In particular, the relative contribution of regressor $j$ is measured by averaging over all possible permutations of $p$ regressors the increase of $R^2$ when regressor $j$ is added to the model based on the  other regressors entered before $j$ in the model.} to evaluate the relative contributions of factor betas and characteristics to the explanatory power of the joint model.  We find that in each factor model, almost 50\% of the model $R^2$ is attributed to the characteristics. The relative contributions of the size and value betas decrease by roughly 50\% when Mcap and B/M are added to the joint model. The RMW and CMA betas contribute roughly twice as much as their respective characteristics.  
\par To sum up, the factor betas and characteristics both account for roughly half of the model explained variation in fund returns. These findings are consistent with those of \citet{chordia2015cross}, which find that in a joint model, firm characteristics explain a majority of cross-sectional variation in stock returns. In the remainder of this paper, we will explore the implications of the equal explanatory power of factor betas and characteristics for the performance evaluation of mutual funds. 

% \begin{singlespacing}
% \begin{table}[h!]
% \setlength{\tabcolsep}{4.5pt}
% \centering
%  {\captionsetup{justification=centering,singlelinecheck=off}
% \caption{\bfseries Cross-sectional regressions of mutual fund returns }}
% \caption*{This table presents the time series averages of risk premia ($\gamma$) estimated using the cross-section of mutual fund (monthly) returns following the \citet{fama1973risk} procedure. The monthly regressions are of the form: 
% \begin{equation*}
% R_{t} = \gamma_{0t} + \gamma_{1t}\hat{B}_{t-1} + \gamma_{2t}Z_{t-1} + \xi_{t}, \hspace{0.2cm} t=1,...,T.
% \end{equation*} We employ the CAPM, the \citet{fama1993common} three-factor model, the \citet{carhart1997persistence} four-factor model, the \citet{FAMA20151} five-factor model and a six-factor model combining all factors. Factor betas ($\hat{B}$) are estimated from rolling time series regressions using daily returns from the past two years. The characteristics (Z) are the logarithm of market capitalization (Mcap), the logarithm of book-to-market ratio (B/M), the logarithm of one plus the cumulative past twelve-month cumulative return (Mom12), operating profitability (Profit) and asset growth (Invest). Each characteristic is winsorized at the 0.5\% and the 99.5\% levels. To address the EIV bias, we employ the EIV-corrected estimator of \citet{chordia2015cross}. Risk premia (in percent per month) are fitted using OLS, both with EIV-correction and without. \citet{fama1973risk} t-statistics are reported in parenthesis. Estimates significant at the 5\% are in bold font. Risk premia are estimated over the period February 2001 until December 2016.}
%  \small
% \label{my-label}
% \begin{tabular}{lrrrrrrrrrrrrrr}
% \hline
%       & \multicolumn{2}{c}{CAPM}        &           & \multicolumn{2}{c}{FF 3FM}      &           & \multicolumn{2}{c}{Carhart 4FM}  &           & \multicolumn{2}{c}{FF 5FM}       &           & \multicolumn{2}{c}{FF 6FM}       \\ \cline{2-3} \cline{5-6} \cline{8-9} \cline{11-12} \cline{14-15} 
%       & OLS            & EIV            &           & OLS            & EIV            &           & OLS            & EIV             &           & OLS            & EIV             &           & OLS            & EIV             \\ \hline
% Cnst  & \textbf{1.013} & \textbf{1.020} &           & 0.685          & 0.628          &           & \textbf{0.887} & \textbf{0.857} &  & \textbf{0.651} & 0.638          &  & \textbf{0.769}  & \textbf{0.759} \\
%       & (2.46)         & (2.49)         &           & (1.88)         & (1.67)         &           & (2.61)         & (2.38)         &  & (2.00)         & (1.93)         &  & (2.47)          & (2.37)         \\
% $\beta_{MKT}$  & -0.071         & -0.080         &           & -0.011         & -0.011         &           & -0.029         & -0.033         &  & 0.130          & 0.177          &  & 0.100           & 0.133          \\
%       & (-0.24)        & (-0.26)        &           & (-0.04)        & (-0.03)        &           & (-0.09)        & (-0.10)        &  & (0.46)         & (0.61)         &  & (0.35)          & (0.46)         \\
% $\beta_{SMB}$ &                &                &           & 0.098          & 0.120          &           & 0.049          & 0.061          &  & 0.099          & 0.099          &  & 0.074           & 0.073          \\
%       &                &                &           & (0.62)         & (0.66)         &           & (0.30)         & (0.33)         &  & (0.66)         & (0.59)         &  & (0.48)          & (0.42)         \\
% $\beta_{HML}$  &                &                &           & \textbf{0.347} & \textbf{0.373} & \textbf{} & \textbf{0.372} & \textbf{0.412} &  & 0.220          & 0.190          &  & 0.266           & 0.252          \\
%       &                &                &           & (2.12)         & (2.09)         &           & (2.04)         & (1.99)         &  & (1.45)         & (1.07)         &  & (1.61)          & (1.30)         \\
% $\beta_{WML}$  &                &                &           &                &                &           & 0.176         & 0.227          &  &                &                &  & 0.114           & 0.195          \\
%       &                &                &           &                &                &           & (1.14)        & (1.19)         &  &                &                &  & (1.01)          & (1.13)         \\
% $\beta_{RMW}$ &                &                &           &                &                &           &                &                &  & 0.190          & 0.224          &  & 0.225           & 0.271          \\
%       &                &                &           &                &                &           &                &                &  & (1.17)         & (1.16)         &  & (1.29)          & (1.32)         \\
% $\beta_{CMA}$  &                &                &           &                &                &           &                &                &  & 0.152          & 0.141          &  & 0.173           & 0.164          \\
%       &                &                &           &                &                &           &                &                &  & (1.13)         & (0.90)         &  & (1.33)          & (1.09)         \\
% Mcap  & -0.065         & -0.065         &           & -0.047         & -0.043         &           & -0.055         & -0.052         &  & -0.050         & -0.051         &  & \textbf{-0.054} & -0.054         \\
%       & (-1.97)        & (-1.97)        &           & (-1.46)        & (-1.24)        &           & (-1.79)        & (-1.56)        &  & (-1.82)        & (-1.77)        &  & (-1.96)         & (-1.89)        \\
% B/M   & 0.004          & 0.003          &           & -0.074         & -0.082         &           & 0.072          & -0.085         &  & -0.025         & -0.010         &  & -0.031          & -0.017         \\
%       & (0.04)         & (0.03)         &           & (-0.98)        & (-1.08)        &           & (-1.08)        & (-1.26)        &  & (-0.36)        & (-0.07)        &  & (-0.50)         & (-0.27)        \\
% Mom12 & \textbf{0.783} & \textbf{0.787} & \textbf{} & \textbf{0.863} & \textbf{0.859} &           & 0.478          & 0.446          &  & \textbf{0.794} & \textbf{0.774} &  & 0.513           & 0.524          \\
%       & (2.11)         & (2.12)         &           & (2.34)         & (2.36)         &           & (1.78)         & (1.67)         &  & (2.17)         & (2.13)         &  & (1.85)          & (1.81)         \\
% Profit    & 0.066          & 0.065          &           & 0.047          & 0.046          &           & 0.019          & 0.016          &  & 0.030          & 0.025          &  & 0.003           & 0.000          \\
%       & (1.18)         & (1.17)         &           & (0.94)         & (0.94)         &           & (0.40)         & (0.34)         &  & (0.68)         & (0.59)         &  & (0.07)          & (0.02)         \\
% Invest  & -0.573         & -0.560         &           & -0.389         & -0.371         &           & -0.391         & -0.355         &  & -0.279         & 0.257          &  & -0.230          & -0.222         \\
%       & (-1.69)        & (-1.67)        &           & (-1.43)        & (-1.38)        &           & (-1.53)        & (-1.40)        &  & (-1.14)        & (-0.97)        &  & (-0.98)         & (-0.87)         \\ \hline
% \end{tabular}
% \end{table}
% \end{singlespacing}
% \vspace{-0.45cm}


% Potential exclusions are small holdings (typically under 10000 shares or \$200.000), cases of confidentiality issues and reported holdings that could not be matched to a master security file. Moreover, the distribution of fund size is heavily right skewed, as the median fund size is below a quarter of the average fund size in most years. The market share of domestic mutual equity funds display a steady rise from about 2\% to 11\% over the full length of the sample. 
% Recently, Wharton Research Data Services released a note on stock splits in the Thomson Reuters database, stating a significant proportion of reported shares held around stock splits are incorrect. In case of a stock split for stock $j$ in quarter $T$, I set the reported shares in quarters $T-1$, $T$ and $T$+1 to missing.
% \subsection{Fund flows}
% \label{flows}
% Mutual fund flows are estimated using monthly data from the CRSP Survivor-Bias-Free U.S. Mutual Fund database. Following the majority of the prior literature on fund flows, I calculate the net capital flow\footnote{\citet{frazzini2008dumb} and \citet{lou2012flow} and  correct for fund mergers when calculating fund flows. As fund mergers are quite rare, this study remains with Eq.(\ref{flow}).} to fund $i$ during month $t$ as
 
% \begin{equation}
% \label{flow}
%     \text{flow}_{i,t} = \frac{\text{TNA}_{i,t} - \text{TNA}_{i,t-1} \cdot (1+\text{R}_{i,t})}{\text{TNA}_{i,t-1}},
% \end{equation}
% where $\text{TNA}_{i,t}$ is fund $i$'s total net assets in month $t$ and $\text{R}_{i,t}$ is the monthly net return of fund $i$ over month $t$. The variable flow reflects the percentage growth of a fund that is due to new investment (under the assumption of dividends being reinvested in the fund). To match flows with quarterly holdings data, I sum monthly flows over the quarter $T$ to calculate quarterly flows. Fund flows are dropped if the percentage difference in TNA in between two months is greater than 200\% or less than -50\%. These extreme flows are rare and are typically related to structural changes within funds, e.g., mergers. 
% \par 
% Numerous studies on fund flows (e.g., \citet{chevalier1997risk},  \citet{sirri1998costly}, and \citet{huang2007participation}) document a convex relation between fund flows and past fund performance. Investors direct their capital to funds with high lagged performance, while tend to stay in funds with deteriorating past returns. This leads to the tournament hypothesis; mutual fund managers are incentivized to take on more risk, as superior returns attract cash flows, while investors do not respond symmetrically to poor returns. In addition, \citet{sirri1998costly} find the performance-flow relation to be most pronounced among funds with higher marketing costs.

% \par 
% I replicate the fund flow predictability analysis (see \citet{sirri1998costly} and \citet{coval2007asset}) by conducting the following regression 
% \begin{equation}
% \label{predflow}
% \text{flow}_{i,T} = c_0 +  \sum^{4}_{k =1} \beta_k  \hspace{0.1cm} \text{flow}_{i,T-k} + c_1 \hspace{0.1cm} \text{alpha}_{i,T-1:T-4} + c_2 \hspace{0.1cm} \text{log}\hspace{0.1cm} (\text{TNA}_{T-1}), 
% \end{equation} 
% where the dependent variable is the fund flow in quarter $T$ on lagged fund performance and lagged fund flows, both measured over the past year. Fund performance is measured by Carhart four-factor alpha\footnote{To calculate monthly fund alpha I conduct rolling-window regressions using monthly fund net returns from the previous twelve months.}. I include fund size (defined as the log of a fund's TNA) measured in quarter $T$-1 as a control variable. Eq.(\ref{predflow}) is estimated using either quarterly fund flows or monthly fund flows, both using up to one year of past flows and fund performance. Regressions using monthly fund flows only cover data from 1992 onwards, as the fund's TNA was reported on a quarterly/annual basis prior to 1992.
% % \footnote{Changes in quarterly TNA can not be too extreme: -0.5 $<$ $\frac{\DELTA \text{TNA}_{i,t+1}} {\text{TNA}_{i,t} }$ $<$ 2}. 

% \par Table 3 reports regression coefficients. Estimated coefficients are in line with prior literature, documenting statistically positive coefficients on past fund performance and past fund flows in the past year. 
% Eq.(\ref{predflow}) can be estimated on the entire database as a pool or by a \citet{fama1973risk} regression. The latter entails cross-sectional regressions in each period $T$, and then calculate the time series average of the coefficients and report t-statistics based on the time series standard error of the mean. The main distinction with the former approach is that Fama-MacBeth estimated coefficients solely focus on explaining cross-sectional differences in flows, whereas a pooled regression coefficients also reflect time series variation in overall flows. As a consequence, the Fama-MacBeth coefficients are more precise and conservative. 

% \subsection{Name changes}
% To identify name changes I use data from the CRSP mutual fund database. I only consider fund data from 1999 onwards, as fund summary data became available on a quarterly basis. A fund name change in quarter $T$ is documented in case the fund name differs from the fund name in quarter $T$-1. I keep track of all the name changes of all the funds in my mutual fund sample, resulting in 6208 name changes. I follow the procedure of \citet{cooper2005changing} to identify name changes relating to a fund's investment style. A fund is assumed to follow an investment style if the fund name contains one of the following style identifiers: "small/sm/smallcap/micro", "large/lg/largecap", "value/val/values", "growth/gr/grth", "momentum/mom", "contrarian/contrafund". A style name change is recorded in two cases: (1) the new fund name contains any of the style identifiers and the old name does not, (2) the new fund name dropped any of the style identifiers. After retaining only style name changes, I am left with 738 name changes that are associated with an apparent change in the fund’s investment
% style as indicated in the fund’s old/new name. Finally, to investigate the change in holdings following a style name-change, I require fund's holdings data before and after the style name- change. This reduces the number of style name-changes to 631. 
% \par Table ..., Panel A summarizes the number of (style) name changes in my sample and records to which style the name change relates to. I record 102 style name changes involving the term "small", 165 involving the term "large", 195 involving the term "value" and 287 involving the term "growth". Untabulated results show only 1 style name change involving the term "momentum" and only 11 involving the term "contrarian". Therefore, the remainder of this paper will exclude style name changes involving "momentum" and "contrarian". Panel B compares fund characteristics of funds subjected to a style name change with the other funds in my database. Fund characteristics are gathered from quarter-end $T$-1 prior to a fund name change to investigate whether funds undergoing a style name-change differ from the rest of the sample. Prior to the style name-change, funds changing their name tend to be older and smaller than funds which do not, and have higher returns over the past year. On both a mean and median level, style name-change funds experience less inflows than their counterparts. Moreover, style name-change funds charge higher expenses and experience higher turnover, which might be explained by these funds altering their existing positions to match the change in fund name.  
% \par Figure 1 graphs the quarterly distribution of style name changes from 1999 onwards. Each panel in the figure illustrates the name changes adopting a certain investment style and dropping the opposite investment style. For instance, Panel A graphs the number of funds with a name change towards a small cap investment style and the number of funds which stop classify themselves as large cap funds. Each panel plots the 3-month lagged hedge portfolio return defined as the return differential between two opposite investment styles\footnote{Hedge portfolios comprise the Small-Minus-Big (SMB) and High-Minus-Low (HML) returns from Ken French's Website. }. As illustrated in Figure 1, most style name changes occurred in the early 2000s, where these changes are concentrated towards the current high-return style. This pattern is less pronounced in the following years, mainly due to the lack of a clear winning investment style. 
% \par 

% \subsubsection{The impact of style name changes on fund flows}
% \citet{cooper2005changing} identify a sample of equity mutual funds that make a style name change over the period 1994-2001 and document an increase in abnormal fund flows to those funds which incorporated the current winning investment style in their name. Using a return-based approach to infer a fund's investment style, they find that 44\% of the funds do not alter their holdings after a style name change. Remarkably, only 22\% of the funds change their holdings to match the investment style they carry out through their new fund name. Remarkably, \citet{cooper2005changing} find no difference in abnormal flows to cosmetic and non-cosmetic style name changes, which seems an indication of investor's irrationality\footnote{The field of research on investor's behavioral biases towards firm names include the work of \citet{lee2001s}, \citet{green2013company}, and \citet{kumar2015s}}. When reviewing the changes in fund characteristics prior to and after the style name changes, they record no improvement in three-factor alphas and lower raw returns. 
% \par This section repeats the work \citet{cooper2005changing} over an extended time range. Section \ref{} discussed prior literature on the determinants of fund flows and the ample evidence for a convex performance-flow relationship. A strong relation is documented between fund flows and both lagged fund flows and lagged fund return. In order to identify abnormal fund flows to style name change funds, I need to control for the drivers of fund flows. The first approach makes use of the model in Eq.(\ref{predflow}). To make sure there is no look-ahead bias, I use coefficients estimated from the cross-section of fund flows in quarter T to predict the fund flow in quarter T+1. Unexpected flows are defined as the difference between the actual fund flow and the predicted fund flow. 
% \par 
% The second method to identify abnormal flows relies on a propensity-score matching method. Propensity-score matching is frequently used in the medical literature (for further reading, see \citet{austin2009some}) and entails creating matching pairs between observations from an experimental group and a control group. Appealing features of this method are the ability to directly compare a high number of characteristics across two groups, and that no restrictions are imposed on these characteristics. The two groups in this analysis are the style name-change funds and the non style name-change funds. Each quarter prior to a style name-change I match each fund from the first group with a fund from the second group using 1-1 matching without replacement, based on the similarity of propensity scores. 
% \par
% Propensity scores are computed as follows. Given a set of fund characteristics in quarter $T$-1, I estimate the conditional probability of a style name change in quarter $T$ using a logit regression. More specifically, 
% \begin{equation}
% \label{prop}
% \text{Propensity score}_{i,T} = \text{P}(Y_{i,T} = 1) =  \frac{1}{1+e^{-z}} 
% \end{equation} 

% \begin{multline}
% \label{logit}
% z = a_{T-1} + \beta_{i,T-1} \hspace{0.1cm} \text{log}(TNA_{i,T-1}) + \beta_{2,T-1} \hspace{0.1cm} \text{Av.} (\text{flow}_{i,T-1:T-4}) + \beta_{3,T-1} \hspace{0.1cm} \text{R}_{i,T-1:T-4} + \\ \beta_{4,T-1} \hspace{0.1cm} \text{Std dev.}(\text{R}_{i,T-1:T-4}) + \beta_{5,T-1} \hspace{0.1cm} \text{alpha}_{i,T-1:T-4}, 
% \end{multline}
% where fund characteristics include the lagged log of TNA, average fund flow over the past four quarters, cumulative monthly fund returns over the past year, standard deviation of past year monthly fund returns and Carhart four-factor alpha measured over the past year. The dependent variable $\text{Y}_{i,T}$ is a dummy variable with value 1 in case of a style name change in quarter $T$ for fund $i$. Parameters in Eq.(\ref{logit}) are estimated by a logit regression for each quarter $T$ in which style name-changes occur. The estimated conditional probability in Eq.(\ref{prop}) is defined as the propensity score. I rank all the style name-change funds according to the propensity score and match each fund to a non style name-change fund with the closest propensity score. Abnormal fund flows for each style name-change are defined as the excess fund flow over the respective matching non style name-change fund flow. 
% \par Table .. Panel A reports raw fund flows, unexpected fund flows and abnormal fund flows to all style name-change funds from four quarter before the name change to four quarters after. Prior to the style name-change, funds do not experience statistically significant abnormal/unexpected fund flows. Funds receive a sharp increase in both raw and abnormal fund flows in the quarter of the style name-change, which slowly declines in the quarters thereafter. The large inflows in the quarter of style name-changes might indicate that the style name-change is publicly available before the quarter-end. Abnormal fund flows remain statistically significant up to three quarters after the style name change before reverting in the fourth quarter. 
% \par All style name-changes are categorized into four categories: style name-changes towards winning vs. losing investment styles and cosmetic vs. non-cosmetic style name-changes. Following \citet{cooper2005changing}, I use size and book-to-market quintile portfolio returns from Ken French's Website as proxies for the investment style premiums. The small cap premium is the return of the bottom 20\% size portfolio; the large cap premium is the return of the top 20\% size portfolio; the value premium is the return of the top 20\% book-to-market portfolio; the growth premium is the return of the bottom 20\% book-to-market portfolio. A style name-change towards (away from) a certain investment style is considered a winning change if the cumulative return of the corresponding style premium over six months prior to the style name-change is positive (negative). In case a style name-change involves multiple style identifiers\footnote{One example is a change from "American AAdvantage Fds: Grth \& Inc/Instl" to "American AAdvantage Funds: American AAdvantage Large Cap Value Fund; Institutional Class". This style name-change adopted the identifier for both large cap and value, and dropped the identifier for growth.}, the net style premium needs to be positive (negative) to be labeled a winning (losing) style name-change. For instance, if the fund name has replaced a large cap identifier by a small cap identifier, the small premium needs to exceed the large cap premium.   