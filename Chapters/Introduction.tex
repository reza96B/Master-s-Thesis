Evalutions of the performance of mutual funds commonly resort to factor models such as the Capital Asset Pricing Model (CAPM), the three-factor model of \citet{fama1993common} or the model of \citet{carhart1997persistence}. Recently, the beta-pricing relation underlying these models has been put under scrutiny, as the empirical asset pricing literature has looked beyond a unilateral explanation of cross-sectional variation in equity returns by factor exposures. Instead, considerable evidence indicates that firm and asset characteristics such as market capitalization, book-to-market ratios, and past returns substantially help explaining the cross section of equity returns, see for example \citet{brennan1998alternative,avramov2006asset,chordia2015cross}. 

The conclusion that both factors and characteristics should be used to explain equity returns implies that both should also be used in the evaluation of the performance of mutual funds and the determination of their alpha. \citet{chen2018characteristics} show that using either factor models or characteristics to adjust performance leads to materially different values for alpha, and a different alpha-based ranking of mutual funds. \citet{busse2017double} propose a double-adjusted alpha as a performance measure. In the first step, returns of a mutual fund are adjusted for factor exposures, and in the second for characteristics of the fund's holdings. They show by means of simulation that their double-adjusted alpha is less prone to errors in performance evaluation than the traditional single-adjusted alpha approach that only adjusts returns for factor exposures.

The improvement in performance evaluation offered by using both factors and characteristics comes at the cost of a more complex analysis. Typically a two-pass procedure is employed, where factor exposures are estimated in the first pass by time-series regressions, and the rewards to factor exposures and characteristics in the second pass by a cross-sectional regression \citep[cf.][]{chordia2015cross,busse2017double}. The pricing errors or alphas, which are taken as measure for skill by \citet{busse2017double} also follow from this second pass. Because the factor exposures are estimated, the second pass suffers from an errors-in-variable bias. \citet{chordia2015cross} show how to correct for this bias, whereas \citet{jegadeesh2016empirical} use an instrumental variable approach to solve it. Using the intercepts of the first pass as dependent variables in a regression on characteristics as in \citet{busse2017double} does not suffer from an errors-in-variables bias, but induces heteroskedasticity and should address that these intercepts are not observed.

In this paper, we show how to circumvent the complexities arising from the two-pass estimation by a hierarchical Bayesian approach to estimating a double-adjusted alpha. The model that we propose for asset returns consists of three layers. In the first layer, we relate daily asset returns to risk factors by beta-coefficients and an intercept that all vary on a monthly basis. This intercept corresponds with the traditional alpha. In the next layer, we relate this monthly intercept to asset characteristics, a time-random effect and an asset-specific component. The sum of the two latter terms gives the double-adjusted alpha. In the third layer, we specify that the time-random effect and the coefficients for the asset characteristics for each month are drawn from a given distribution with constant mean and variance. The asset-specific component is drawn from a distribution with a zero mean and a variance that varies per month. Our model is similar to \citet{cederburg2015asset}, though we allow for more risk factors in the first layer. Moreover, they combine monthly returns with yearly varying coefficients. \citet{cosemans2015estimating} use a similar setup to model time-variation in betas related to firm characteristics.

As in \citet{busse2017double}, our double-adjusted alpha is constructed by adjusting for exposure to risk factors and the effect of characteristics, but our approach comes with several advantages. First, we can estimate it in a single pass, which leads to efficiency gains and inferences for the double adjusted alphas that account for the estimation uncertainty from both the time series and the cross-sectional domain. Second, when estimating the parameters for mutual fund returns, the time-random effect in the double-adjusted alpha measures the out- or underperformance of the average mutual fund in a particular month. The mean of the distribution from which the time-random effects are drawn measures the extent to which the average mutual fund can deliver outperformance. We call this mean the aggregate alpha. Third, because we use a Bayesian approach, we can specify a prior distribution for it. For example, if we believe that the model uses the correct risk factors and characteristics, and that the average mutual funds does not possess skill, the prior distribution of the aggregate alpha should be centered on zero. The posterior distribution then shows the support for this hypothesis in the data.

We use the hierarchical Bayes approach to construct double-adjusted alphas for actively managed equity mutual funds in the U.S. over the period 2001--2016. We take the risk factors from the five-factor model of \citet{FAMA20151}, augmented with a momentum factor as in \citet{carhart1997persistence}. \citet{jordan2016skill} provide evidence favoring the addition of the profitability and investment factors next to the standard market, size, value and momentum factors. We also include the corresponding characteristics: marktet capitalization, book-to-market ratio, momentum, operating profitability and asset growth. We aggregate the firm characteristics of a fund's holdings to compute aggregate monthly fund-level characteristics. Similar to the evidence on stock returns, we find that mutual fund factor betas and holdings characteristics are correlated, but the correlations are modest in magnitude (e.g., the average of the absolute value is roughly 0.4), implying that factor betas and characteristics do not convey identical information. When we conduct monthly cross-sectional regressions of mutual fund returns on factor betas and characteristics as in \citet{avramov2006asset}, we find that both account for about half of the model explained variation. This highlights the importance of including both and constructing double-adjusted alphas.

Our results confirm that characteristics should be taken into account when evaluating mutual fund performance. The posterior distribution of the coefficient for each characteristic when included in isolation strongly indicates values different from zero, except for profitability. Including all characteristics simultaneously yields similar estimates to the univariate results. This result means that funds can exhibit higher relative performance based on standard factor model alpha by passively loading on characteristics, even when the factor model explicitly adjust returns for those characteristics. When we use the frequentist two-pass procedure as in \citet{busse2017double}, we do not find an effect for the size, momentum and investments characteristics. In our Bayesian approach, the posterior mean of the aggregate alpha is negative, and the 95\% posterior density does not include zero. The two-pass procedure yields estimates that are close to this posterior mean, but it's not significant, indicating that our Bayesian approach leads to higher precision.

Next, we analyze the impact of our double-adjusted performance measure on relative mutual fund performance. We find that the median change in percentile performance ranking is roughly 9\%. That is, a fund ranked in the median percentile according to the traditional six-factor alpha would be ranked in the 41th or 59th percentile based on our double-adjusted measure. Furthermore, we find that a large number of funds exhibit dramatic changes in percentile ranks, with 10 (5) percent of funds exhibiting a mean change in percentile ranking greater than 22.99\% (27.24\%). 

We continue our analysis by showing that replacing the traditional alpha by our more refined double-adjusted alpha has consequences for our understanding of the causes and consequences of skill in the mutual fund industry. First, we show that the changes in relative performance alter inferences on the persistence of mutual fund performance as documented in \citet{carhart1997persistence,bollen2004short}. We create decile portfolios of mutual funds ranked on their traditional alpha and their double-adjusted alpha based on the past 24 months. In both cases we observe that the top decile beats the bottom decile in terms of alpha over periods of one quarter, one year and three years. However, when we rank on double-adjusted alpha, significance levels are higher with $t$-statistics of the difference well over 5, the difference is also present in returns, and persistence lasts up to six years after portfolio formation. These results confirm the findings of \citet{busse2017double}, though we examine a shorter period. The part of the traditional alpha that is related to characteristics does not predict differences in returns or alpha, showing that that part of the traditional alpha should be excluded from a measure of skill.

Second, we show that skill can no longer be linked to the selectivity of a fund. \citet{amihud2013mutual} document that the $R^2$ of the first-pass regression of fund returns on the returns to factor portfolios is negatively related to alpha. They argue that low $R^2$ values show that fund managers deviate from factor portfolios, and interpret the relation with alpha as a sign of skill. We find that low $R^2$ values are related to the part of the traditional alpha that is related to characteristics, but not to the double-adjusted alpha that is a better measure for skill. It means that as commonly found in finance literature low values of $R^2$ indicate that factor models do not explain returns well, but they cannot be directly interpreted as a measure for selectivity related to skill. Our results are a bit stronger than those reported by \citet{busse2017double} over a longer time sample period.

Third, we investigate the relation between fund flows, factor exposures and alpha as in \citet{barber2016factors}. They argue that truly sophisticated investors should adjust the returns of mutual funds for all factor exposures and effects of characteristics, and should select funds only based on their true alpha. In their analysis they only take factor exposures into account, but do not investigate the effect of characteristics. We complement their analysis by showing that fund flows are most responsive to the double-adjusted alpha. The effect of the double-adjusted alpha on fund flows is ten times larger than the part of alpha that is related to characteristics. The effect of the characteristic part is mainly driven by size and momentum. We interpret this result as evidence that at least part of the investment community can really distinguish skilled fund managers.

Our findings contribute to the literature in two ways. First, we show how to efficiently estimate double-adjusted alphas based on a hierarchical Bayesian model that combines both time series and cross-sectional information, but without the need for a two-pass procedure. We theoretically argue and empirically show that our approach leads to more precise estimates of the double-adjusted alpha and the effect of characteristics than the two-pass approach that is used by \citet{busse2017double,chordia2015cross}.

Second, we confirm and extend the conclusion of \citet{busse2017double} that the effect of characteristics present in traditional alphas contaminates further analyses related to mutual fund performance. Conform their results, rankings of mutual funds change substantially, leading to stronger evidence of persistence in mutual fund performance. We also find that the relation between time series $R^2$ and skill disappears. As an extension, we show that the relation between alpha and fund flows survives when we use double-adjusted alphas. Taken together, we conclude that the method to measure alpha is crucial in the analysis of skill in mutual funds.

This paper proceeds as follows. Section 2 describes the data set, the construction of the fund-level characteristics, and the cross-sectional regressions of mutual fund returns on factor betas and characteristics. Section 3 describes our hierarchical Bayes approach to obtain our double-adjusted performance measure. Section 4 examines the implications of this new performance measure on previous findings in the mutual fund literature. Section 5 contains robustness checks. Finally, Section 6 concludes.

\begin{comment}
\subsection{old}


Most studies on the performance evaluation of mutual funds resort to factor models such as the Capital Asset Pricing Model (CAPM) or the model of \citet{carhart1997persistence}. Recently, the beta-pricing relation underlying these models have been put under scrutiny, as the empirical asset pricing literature has looked beyond a unilateral explanation of cross-sectional variation in returns by factor exposures (betas). Instead, considerable evidence indicates that individual firm characteristics such as market capitalization, book-to-market ratios, and past returns are better predictors of expected asset returns. This calls into question the usage of standard factor models to evaluate mutual fund performance.  

The drivers behind asset returns have been a staple of modern finance. The CAPM has become a pivotal model in asset pricing theory, explaining asset returns solely by the exposure to the market. Later, considerable evidence of cross-sectional patterns (so-called anomalies) in asset returns raised doubts about the CAPM. Size \citep{banz1981relationship} and book-to-market \citep{fama1992cross} effects arose; returns on stocks with a small market capitalization have historically exceeded returns on stocks with a large market capitalization, and high book-to-market (value) stocks displayed better performance relative to low book-to-market (growth) stocks. \citet{jegadeesh1993returns} document price momentum, as past winning stocks show strong abnormal performance relative to past losing stocks. Moreover, recent literature find high abnormal returns obtained by high quality stocks.\footnote{Quality is measured by accruals in \citet{bender2013earnings} and by profitability, stable growth, and a high payout ratio in \citet{asness2014quality}.} 

Several prominent studies have proposed multi-factor models, which incorporate risk factors that can parsimoniously account for the aforementioned anomalies simultaneously. In particular, \citet{fama1993common} advocate a three-factor model that includes risk factors adjusting for size- and value-effects. The size and value factors are the excess returns on the factor-mimicking portfolios for market capitalization (small minus big, SMB) and book-to-market (high minus low, HML), respectively. \citet{carhart1997persistence} adds the momentum factor, WML, which is the return spread between past winning stocks and past losing stocks. \citet{FAMA20151} have proposed the profitability (robust minus weak, RMW) and investment (conservative minus aggressive, CMA) factors. 

Despite the widespread use of these multi-factor models, we beg the question whether the risk factors fully account for the anomalies underlying these factors. Recent studies have shown that stock returns with high factor betas do not necessarily imply high values for the corresponding firm characteristics underlying these factors.\footnote{\citet{chordia2015cross} and \citet{chen2018characteristics} both find that the cross-sectional correlations between factor betas and underlying characteristics of stocks are below 0.5 in absolute value.} Similar to the evidence on stock returns, we find that mutual fund factor betas and holdings characteristics are correlated, and the correlations are modest in magnitude (e.g., the average of the absolute value is roughly 0.4), implying that factor betas and characteristics do not convey identical information. Moreover, \citet{brennan1998alternative} find that characteristics explain the cross-section of risk-adjusted returns (alphas), while \citet{busse2017double} reach the same conclusion using the returns of mutual funds. We conduct monthly cross-sectional regressions of mutual fund returns on factor betas and characteristics\footnote{We aggregate the firm characteristics of a fund's holdings to compute aggregate monthly fund-level characteristics.} and find that both account for about half of the model explained variation. 

This paper applies previous findings of both factor betas and firm characteristics explaining cross-sectional variation in stock returns to the returns of mutual funds. In the context of mutual fund performance we propose a novel performance measure that controls for both types of exposures. Specifically, we decompose alpha into two components: (1) characteristic-driven performance, the component which is associated with passive loadings on firm characteristics and (2) our double-adjusted performance measure, which we define as the difference between alpha and characteristic-driven performance. To do this, we use a hierarchical Bayes approach to simultaneously estimate the conditional factor model parameters for each fund using daily returns in a rolling-window scheme, and the monthly cross-sectional relation between conditional alphas and firm characteristics. The main advantage of our approach is the simultaneous estimation of the model which mitigates the measurement error problem the traditional two-pass procedures are prone to (e.g., \citet{brennan1998alternative}, \citet{avramov2006asset} and \citet{busse2017double}).  

To understand the importance of adjusting fund performance to characteristics, consider the following example. Assume that we estimate the Carhart four-factor model for a certain fund and obtain an estimated monthly alpha of 10 basis points. Now consider that this fund follows a momentum strategy, and further suppose that it is common knowledge that in the universe of mutual funds, the momentum premium earned by funds exceeds that projected by the WML factor (see, \citet{huij2009use}). This implies that the momentum factor under-adjusts for momentum effects, such that fund managers can generate alpha by simply taking positions in high momentum stocks. That is, the four-factor model ``falsely'' awards superior skill to funds following a momentum strategy, which is not founded on special information or skill to exploit mispricing. Thus, it is important to properly adjust for exposures to these investment styles to evaluate the fund manager's true skill and, in turn, managerial compensation. 

% \par In preliminary analysis we use the net shareholder returns for a sample of actively managed U.S. domestic equity mutual funds to run monthly cross-sectional regressions (CSRs) of mutual fund returns on factor betas and characteristics. We aggregate the firm characteristics of a fund's holdings to compute aggregate monthly fund-level characteristics. The factor betas are estimated from the CAPM, the Fama-French three- and five-factor models, and models which added the momentum factor to the Fama-French factors. Since factor betas are estimated with error, the CSRs are inherently subject to the errors-in-variable bias (EIV bias), which leads to a negative bias in estimated risk premia. We address the EIV bias with the EIV-corrected estimator of \citet{chordia2015cross}. The cross-sectional regressions provide evidence that market capitalization and momentum return yield economically significant coefficients, even in a joint model with the size and momentum betas. The estimated risk premia of all factor betas decrease in magnitude when their underlying characteristics are added to the model. Moreover, we find that factor betas and characteristics both account for about half of the model explained variation.

We use the hierarchical Bayes approach to adjust alphas for five characteristics over the period 2001 to 2016: market capitalization, book-to-market, momentum, operating profitability, and asset growth. When we examine the relation between conditional Fama-French six-factor alphas and each characteristic in isolation, we find that each characteristic except operating profitability is significantly related to alphas. Considering all characteristics simultaneously we find that estimates are similar to the corresponding univariate estimates. When we estimate the relation between alphas and characteristics using the traditional OLS two-step procedure, we find insignificant relations between alphas and the size and momentum characteristics. To ensure the robustness of our results, we estimate the Bayesian model for the other multi-factor models. The results are qualitatively similar across factor models, such that we find statistical significance for all characteristics except operating profitability. Thus, funds can exhibit higher relative performance based on standard factor model alpha by passively loading on characteristics, even when these factor models explicitly adjust returns for those characteristics. 

We analyze the impact of our double-adjusted performance measure on previous studies examining relative mutual fund performance. To provide an indication of the impact of the characteristic-adjustment of alpha on relative performance, we find that the median change in percentile performance ranking is roughly 9\%. That is, a fund ranked in the median percentile according to six-factor alpha would be ranked in the 41th or 59th percentile based on our double-adjusted measure. Furthermore, we find that a large number of funds exhibit dramatic changes in percentile ranks, with 10 (5) percent of funds exhibiting a mean change in percentile ranking greater than 22.99\% (27.24\%). 

One can anticipate that changes in relative performance of this magnitude might alter inference regarding relative mutual fund performance, which in the majority of past literature has been based on Fama-French alphas. Central to the mutual fund literature are studies on persistence in fund performance (e.g., \citet{carhart1997persistence}, \citet{bollen2004short}). When we adjust performance for both factor exposure and characteristics, we find that our new performance measure predicts post-ranking performance up to six years. In line with the results of \citet{busse2017double}, the double-adjusted performance measure reflects true skill, which persists in the long run such that the documented persistence in standard alpha is mostly attributable to the component of alpha corrected for loadings on characteristics. In addition to performance persistence, past literature on relative fund performance also includes numerous studies which examine the relation between fund performance and a specific fund feature, such as a fund's factor model R-squared \citep{amihud2013mutual}, or the capital flow of a fund \citep{barber2016factors}, among others. We replicate parts of the analysis of these two studies using our decomposition of alpha as performance measures. We find that the inverse relation between fund performance and R-squared found in the original study is mostly driven by the characteristic-driven component of alpha. Similarly, we find that components of alpha driven by size and momentum characteristics are significant predictors of future fund flows, suggesting that investors tend to these characteristics when assessing fund performance. 

Our paper has a niche in the intersection between the asset pricing literature and the mutual fund performance literature. \citet{fama1993common} and \citet{davis2000characteristics}, among many others, argue that it is factor betas that explain expected returns, while \citet{daniel1997measuring} contend that it is characteristics.  \citet{brennan1998alternative} is the forerunner in the literature which considers both factor betas and characteristics explaining returns. Their main finding was that characteristics such as market capitalization and book-to-market ratio explained deviations from the Fama-French three-factor model. This study paved the way for others, including the work of \citet{chordia2015cross}, which was the first study to directly evaluate the fraction of explained variance of both factor betas and characteristics in a joint model. Inspired by their work are the cross-sectional regressions in this paper, however, we know of no study which analyze the role of factor betas and characteristics in explaining the returns of mutual funds in a joint model. Our analysis shows that previous findings of firm characteristics explaining a large fraction of the variation in expected stock returns can be extended to the returns of mutual funds. 

The closest work to this paper is \citet{busse2017double}, who construct a performance measure adjusting for both factor betas and  characteristics, and study the implications of their double-adjusted measure on previous findings in the mutual fund literature. The setup and goal of this paper are similar, but the execution differs in a number of ways. The first part of our paper serves as a theoretical foundation for the characteristic-adjustment of alpha by examining the relation between fund returns and characteristics in greater detail. Moreover, our paper extends the double-adjusted performance measure with the profitability and investment factors. \citet{jordan2016skill} advocate the use of these quality factors as they uncover more distinct patterns in mutual fund performance which are masked when using three- or four-factor alphas. Most importantly, our Bayesian estimation does not only circumvent the measurement error problem from the two-step OLS procedure in  \citet{busse2017double}, it also increases precision of alpha estimates by exploiting cross-sectional information. Finally, our Bayesian estimation benefits from combining data sampled at different frequencies, similar to the hybrid estimator proposed by \citet{cosemans2015estimating} for modelling conditional factor betas.\footnote{An attractive feature of both our models is the use of daily returns to obtain rolling estimates from a multi-factor model and the use of cross-sectional data at a monthly frequency. Apart from the fact that \citet{cosemans2015estimating} condition factor betas rather than alpha on a set of conditioning variables, they also specify a time-invariant relation between the factor betas and conditioning variables, whereas we allow the relation between alpha and characteristics to vary across time.} By using daily returns, we obtain more precise rolling factor beta estimates than those used in the majority of previous studies which are based on monthly returns.

This paper proceeds as follows. Section 2 describes the data set, the construction of the fund-level characteristics, and the cross-sectional regressions of mutual fund returns on factor betas and characteristics. Section 3 describes our hierarchical Bayes approach to obtain our double-adjusted performance measure. Section 4 examines the implications of this new performance measure on previous findings in the mutual fund literature. Section 5 contains robustness checks. Finally, Section 6 concludes.
%There is a vigorous debate about the drivers behind these anomolies. One stream is the behavioral finance literature, which points to psychological biases among investors, while a second stream supports a risk-return relation, stating that these anomolies are rewards for exposures to systematic risk.

% \par The importance of a fund's adopted factor investment style goes beyond the capability of a fund to capture the factor premiums. The degree of volatility in the fund's portfolio style characteristics can be a substantial source of risk for those who invest in the fund. Managers maintaining a more stable investment style may lead to lower turnover and expenses and thus higher returns. A less volatile implementation of style strategy might also signal superior skill as opposed to funds with a higher degree of style volatility, which can be perceived as opportunistic funds. On the other hand, higher style drift might be a result of a manager's superior style timing ability. Closely related to fund performance is fund investors behavior in response to the magnitude of style drift. The relation between fund flows and style volatility is ambiguous for the above mentioned reasons. It is interesting to see whether style volatility contributes to the widely documented relation between fund flows and fund performance. 
% \par There are three ways to identify a mutual fund's investment style: fund name or fund investment prospectus, a return-based approach and a holdings-based approach. In the investment prospectus, funds provide information on their investment style, strategy and philosophy. Investors use this information to indicate different return-risk profiles across funds.\footnote{The Investment Company Institute (ICI) conducted a survey showing 40\% of retail investors consult the fund investment prospectus to identify a fund's investment style. In 2001, the US Securities and Exchange Commission (SEC) tightened the rules concerning the investment objective, stating mutual funds should hold a minimum of 80\% of the portfolio in assets corresponding to the fund's investment style. }. The second approach is based on fund returns. This approach extracts style loadings from factor models by conducting a regression over a historic return window. This method is easy to apply but comes with a few flaws. Firstly, the relation between a fund's net return and factor investment style might be distorted by charged fund fees. Returns-based analysis also fails to capture abrupt shifts in invesmtent style, as the regression window requires a suffficient number of observations.
% \par
% This study investigates a fund's incorporation of three distinct investing styles by collecting the following characteristics of fund's portfolio holdings: the market capitalization (size), the ratio of book-equity to market-equity (value), and past year cumulative return (momentum). The underlying reason to focus on the aforementioned factors is that these three factors are deeply ingrained in prior literature on factor investing, having been studied for decades as part of the asset pricing literature and the practitioner risk factor modeling research. Factor investing has gained acknowledgement under both institutional and retail investors. It entails harvesting the well documented factor premiums which have exhibited excess return above the market. \citet{basu1977investment} and \citet{banz1981relationship} are two pioneer studies which relate price-earnings ratios and market capitalization to stock returns. The work of \citet{fama1992cross}, which presents a multi-factor asset pricing model that accounts for these attributes, highlighted the explanatory power of these variables in the cross-section of equity returns. \citet{jegadeesh1993returns} led one of the first seminal studies on momentum in the US stock market. \citet{carhart1997persistence} incorporated the momentum factor in the Fama-French Three Factor Model leading to the Four-Factor model, which has become a canon within finance literature. 

% According to the Investment Company Institute, the aggregate capital flows (both purchases and redemptions) to all U.S. equity funds was approximately \$47 billion in the first quarter of 2016


% \citet{bender2013foundations} provide a thorough summary of the documented literature on the aforementioned factor premiums, reviewing both the existence and the proposed drivers of these premiums. 
% and the low-volatility (or low-beta) premium\footnote{Later reviewed by \citet{blitz2007volatility}.}  \citep{jensen1972capital}. More recently, the dividend yield premium\footnote{\citet{blume1980stock} and \citet{litzenberger1979effect} found a positive relation between the risk-adjusted returns the expected dividend yield. The work of \citet{black1974effects} and \citet{ang2007stock} contradict the existence of the yield premium, stating the excess return predicted by dividend yield is not robust.  } and the quality premium\footnote{Recent literature relate high excess returns to high quality stocks. Quality is measured by accruals in \citet{bender2013earnings} and by profitability, stable growth and a high payout ratio in \citet{asness2014quality}.} have become increasingly well-accepted in the academic literature. 

% The world has witnessed an explosive growth in assets of U.S. equity mutual funds; the current total assets have almost tripled relative to the beginning of this decade.
% Mutual fund flows played a central role in many existing studies, which can broadly be categorized into those who investigate the drivers of these capital flows and the stock price pressure exerted by these capital flows. This paper falls into the latter category. 
% \par In this paper I extract mutual fund trading induced by capital flows and investigate its impact on individual stock prices. There is an extensive body of literature on the price impact of institutional flows on stock returns. \citet{warther1995aggregate} reports a positive relation between aggregate capital flows into equity funds and 
% contemporaneous stock returns. \citet{edelen2001aggregate} studies the relation between market returns and aggregate fund flows on a daily basis and document a positive association between aggregate daily flow and concurrent market returns. They also state that aggregate flows follow market returns with a one-day lag. More recently, \citet{boyer2009investor} confirm the positive contemporaneous correlation between flows and stock returns and document a negative correlation between flows and subsequent stock returns, indicating a reversal of prices to fundamental levels after the price pressure has been absorbed. 
% \par Instead of focusing on aggregate fund flows, I examine the individual stock price pressure caused by mutual fund flow-induced trading. Following the methodology of \citet{lou2012flow}, I extract the fund flow-induced trading component from total fund trading. Using this flow-induced component I create a measure of flow-induced price pressure (FIPP) for each individual stock, and link this measure to the well-known empirical findings on stock return anomalies (size, value, momentum). Put more formally, I analyze the potential role of FIPP in explaining some of these well-known factor premiums, by incorporating the FIPP measure as an explanatory variable alongside measures of these factors in \citet{fama1973risk} regressions to forecast future stock returns. In order to disentangle the FIPP effect from the other factors, I apply a double-sorting routine, first on one of the traditional factors and subsequently on FIPP. Finally, I construct a hedge portfolio based on FIPP and include this factor in an asset pricing model of the likes of the three factor model of \citet{fama1993common} and the four factor model of \citet{carhart1997persistence},  explaining the cross-section of returns. 
% \par The existence of factor premiums is widely acknowledged such that literature has shifted its attention towards an explanation for their existence. Understanding the drivers behind these factor premiums is crucial to make any inference on their persistence and to incorporate these factors in an investment strategy. Current explanations for the well documented factor premiums momentum and value can be summarized in two separate views- one based on market efficiency where factor premiums are a compensation for bearing a certain risk, and one based on behavioral biases among investors. Despite an extensive amount of research, academic literature is yet to reach a widely supported consensus on the drivers behind factor premiums. With this paper I aim to contribute to the ongoing debate in factor investing. 
% \par The flow-based mechanism mostly relates to the well documented momentum factor \footnote {\citet{jegadeesh1993returns} led one of the first seminal studies on momentum in the U.S. stock market, showing significant outperformance of past winning stocks over past losing stocks in 1965-1989, where past performance was measured over the past 3-12 months. \citet{carhart1997persistence} expanded the three factor model of  \citet{fama1993common} with a momentum factor as an additional variable. \citet{fama2012size} investigated stock price momentum on a global level, and found strong momentum returns everywhere except for Japan. } \citep{jegadeesh1993returns}. The momentum factor is the tendency of stocks with good recent performance to outperform stocks with bad recent performance in the near future. The theory underlying momentum is still a matter of debate. Unlike the size factor \citep{banz1981relationship} and the value factor \citep{fama1992cross}, there is no satisfactory explanation within a market-efficient framework. A risk-based explanation for the momentum premium has been dismissed by several studies\footnote{See \citet{fama1996multifactor},  \citet{moskowitz2003analysis} and \citet{liu2004economic} }
% showing that the momentum premium is not a compensation for bearing risk. The prevalent explanations for momentum are derived from behavioral models, and assume investors misinterpret information signals and behave accordingly. Investors either over-react \citep{barberis1998model} or under-react \citep{hong2000bad} to stock-specific news, both leading to the momentum effect. 
% \par Recently, a third explanation arose which is centered around the dynamics of institutional investing rather than investor biases. This stream of literature includes the work of \citet{vayanos2013institutional}, in which they state that momentum and reversal effects arise by capital flows moving in and out of mutual funds. When a negative shock hits the fundamental value of certain assets, the returns of investment funds holding these assets decline, leading to outflows by investors who observe the deterioration in the performance of the fund. Funds react to these outflows by lowering the positions in the assets they own, which in turn further depresses the prices of the assets hit by the original shock. Their paper argues these outflows to be gradual (due to lock-up periods, institutional decision lags), such that the selling pressure causes to decline gradually, causing the momentum effect. Moreover, since these outflows push prices below the fundamental value of the assets, we expect prices to revert eventually, leading to reversal effects. They formalize this intuition by constructing an infinite-horizon continuous-time model with fully rational agents, and they show momentum effects still arise despite investor's rationality. 
% \par Closely related to reversal is the value factor, which is the tendency of stocks with a low market valuation relative to book value to outperform in subsequent periods. Prior literature propose several drivers behind the value factor. \citet{cochrane1996cross}, \citet{zhang2005value} and \citet{risk2012macro} suggest that value firms, in contrast to their growth counterparts, are less adaptable to unfavorable economic environments. The value premium can thus be viewed as a compensation for macro risk. Recently, \citet{de2011value} contradict a risk-based interpretation of the value anomoly, documenting no relation between the value premium and proxies for distress risk.  \citet{lakonishok1994contrarian} and \citet{barberis2001mental} propose several behavioral biases that may explain the value factor. A flow-based driver is investigated by
% \citet{frazzini2008dumb}, in which they construct a mutual fund database over 1980-2003 and show that retail investors direct their money to funds which hold stocks with poor future performance, also labeled the "dumb money" effect. They take a closer look at the relation between the dumb money effect and the value factor, by performing a double-sort of all stocks on both fund flows and book-to-market ratios. Their results indicate that neither effect dominates the other, suggesting a common driver behind both effects. 
% \par The closest work to mine is \citet{lou2012flow}, in which the flow-based mechanism is related to a number of well-known empirical findings on return predictability. They state that the flow-based mechanism fully explains both mutual fund persistence \footnote{\citet{grinblatt1992persistence} and \citet{carhart1997persistence} are examples of papers documenting significant persistence in the risk-adjusted performance ranking across mutual funds.}, the smart money effect \footnote{\citet{gruber1996another}, \citet{zheng1999money} and \citet{keswani2008money} report that investors channel money toward mutual funds that subsequently perform well, also dubbed the "smart money" effect. \citet{sapp2004does} argue that the smart money effect is completely sublimed after correcting for stock price momentum.} and it partially explains the momentum effect. In a Fama-MacBeth regression to forecast future stock returns, the inclusion of expected flow-induced price pressure next to lagged cumulative stock returns reduces the coefficient of the latter with 25\% to 42\%. They hypothesize that winning mutual funds scale up their existing holdings with capital inflows that are (by definition) concentrated in past winning stocks, driving up the price of past winning stocks in subsequent periods. In contrast, losing mutual funds scale down existing holdings that are concentrated in past losing stocks, driving down the price of past losing stocks in subsequent periods. They report a positive return spread between the top and bottom FIPP deciles in the ranking quarter, which is indistinguishable from zero in the subsequent year and completely reversed in years two and three after the ranking. This result contradict  prior literature on mutual fund flows (e.g., \citet{coval2007asset} and \citet{frazzini2008dumb}), which document an immediate reversal after capital flows. 

% flows into mutual funds (and thus into their stock holdings) can explain up to 50\% of the momentum effect. Winning funds attract capital inflows, such that these funds scale up their holdings, which are (by definition) among the winning stocks. This implies that investors are buying winner stocks from the previous period by shifting their resources to winning funds. They construct a variable representing the aggregate flow-induced trading of a stock from all mutual funds and through \citet{fama1973risk} regressions show that the explanatory power of the momentum effect drop significantly after controlling for flow-induced trades. The key feature of a flow-based driver behind factor premiums is the interplay between stock returns and mutual fund returns. Expected stock returns are partially driven by the past performance of mutual funds which hold a position in the stock. Similarly, the expected performance of funds are partially driven by the past performance of all other mutual funds that are holding overlapping positions. 

%  Current explanations are summarized in two separate views - one based on market efficiency where factor premiums are a compensation for bearing a certain risk, and one based on behavioral biases among investors. An example of the first view is the size premium being a compensation for exposure to small firms which are less liquid \citep{amihud2002illiquidity} or contain a high default probability \citep{vassalou2004default}. An example of the second view is the stream of literature which relate the momentum premium to investors\textsc{\char13} behavioral biases. \citet{barberis1998model} and \citet{daniel1998investor} propose behavioral models which relate the momentum premium to investors under-reacting to new information and slow information diffusion by financial markets. 

\end{comment}