We propose a new approach to evaluate the performance of mutual funds. It combines traditional factor models as in \citet{carhart1997persistence,fama1993common} with the effects of firm and asset characteristics, conform the recent insight that these information sources complement each other in explaining asset returns \citep{avramov2006asset,brennan1998alternative,chordia2015cross}. \citet{busse2017double} shows that the firm and asset characteristics resulting from mutual fund holdings should also be included in analyses of mutual fund performance.    

Our approach consists of a hierarchical Bayesian model with three layers in the style of \citet{cederburg2015asset}. The first layer specifies a factor model for daily returns, whereas the second splits the intercept in the effect of firm and asset characteristics, and the remainder which we call the double-adjusted alpha. The third layer adds monthly time variation to it. The double-adjusted alpha for each asset and each point in time can be split in an aggregate time-random effect and a time- and asset-specific component. Though our model combines the same information as \citet{busse2017double} to construct the double-adjusted alpha, the Bayesian approach offers a couple of advantages. Whereas \citet{busse2017double} uses a two-pass estimation, our Bayesian approach estimates the time series and cross-sectional parts of the model in one pass which leads to efficiency gains and  automatically accounts for the estimation uncertainty that otherwise should be included in the second part. More generally, the posterior mean of the time-random component of alpha shows to which extent the aggregate mutual fund can deliver outperformance, while the prior can be used to specify the belief in outperformance. 

We apply our approach to analyse the performance of U.S. mutual funds over the period 2001--2016, with the risk factors from \citet{FAMA20151} and a momentum factor, as well as the corresponding characteristics. We show that both factor exposures and the effect of characteristics are needed to explain mutual fund performance, and that our Bayesian approach leads to more precise estimates than the frequentist approach. Moreover, we show that the ranking of mutual funds based on the posterior mean of the double-adjusted alpha differs substantially from the ranking that results from the traditional alphas based on factor models.

We then show how our new inferences on mutual fund performance impact research into the skill of mutual funds. Replacing the traditional alpha by our double-adjusted one leads to stronger evidence for persistence in the performance of mutual funds. However, when we use the double-adjusted alpha, we do not find a relation with the time-series $R^2$ of the factor regression anymore as in \citet{amihud2013mutual}, indicating that the $R^2$ cannot be interpreted as a measure of selectivity related to skill. We extend \citet{barber2016factors} by showing that fund flows are most responsive to the double-adjusted alpha, and not so much to the part that is related to characteristics. We interpret this as evidence that at least part of the investment community also adjusts returns for the part that can be explained by well-known return drivers.

Our findings contribute to the literature in two ways. First, we show how to efficiently estimate a double-adjusted alpha in a single-pass using Bayesian techniques. We argue theoretically and show empirically that our approach improves upon the more standard two-pass estimation methodology by offering better and more precise estimates. Second, we show that these improved inferences are relevant for studies related to the performance and skill of mutual funds. Some of the results from earlier studies turn out to be stronger, while other results vanish. Taken together, it means that we have to be careful in the investigation of mutual fund performance, and should make sure that we include the necessary information in the right way. Our paper shows how a straightforward Bayesian model can accomplish that.

\begin{comment}
\subsection{recent}
Growing concerns arise among academics and practitioners alike regarding the portfolio-based approach for identifying anomalies and, more generally, for testing asset-pricing models. This approach sorts stocks on a firm characteristic and constructs a zero-cost hedge portfolio which reflects the return differential between stocks in the top characteristic decile and stocks in the bottom characteristic decile. \citet{ang2008using} and \citet{fama2008dissecting}, among many others, argue that grouping stocks into portfolios and aggregating returns wastes and potentially distorts cross-sectional patterns in stock returns. Consequently, the Fama-French multi-factor models, that incorporate these portfolios as risk factors, do not fully adjust returns for the main anomalies. Instead of estimated factor betas many studies (e.g., \citet{avramov2006asset}, \citet{chordia2015cross}) have opted for individual firm characteristics to explain expected returns.
\par Our paper moves the debate of factor betas versus characteristics to the returns of mutual funds. Specifically, we incorporate both in the performance evaluation of mutual funds by constructing a performance measure that adjusts returns for both factor betas and characteristics simultaneously. While it is debatable whether fund managers that actively shift their portfolios towards a certain characteristic dimension qualifies as skill, we believe that a performance measure should fully adjust for an anomaly. \citet{busse2017double} was the first to propose such a double-adjusted alpha which is calculated in a two-step procedure. We propose to estimate a double-adjusted alpha by a hierarchical Bayesian approach, which poses several advantages over the approach in \citet{busse2017double}. These advantages mainly stem from the simultaneous estimation of the performance measure, which circumvents using estimates of alpha in a second-pass regression. 
\par Similar to \citet{busse2017double}, we find that characteristics are statistically significantly related to fund returns even after controlling for factor model betas, suggesting that performance identified via Fama-French factor models could be attributed to loadings on characteristics. When we reassess several previous studies on mutual fund relative performance, we find that our double-adjusted performance measure is a purer measure of skill when we analyze persistence in fund performance. Moreover, we find that relations between specific fund features (fund selectivity and fund flows) and fund performance is partially driven by the characteristic component of performance. All in all, by more fully adjusting performance for the influence of characteristics we might alter previous inference on relative mutual fund performance. 
\par Our hierarchical Bayes model can be extended in various directions. For example, the MIDAS approach of \citet{ghysels2005there} can be adopted to assign optimal weights to the daily returns in the window used to estimate the rolling window factor betas. In addition, one can impose a common structure on the factor betas too, rather than just the alphas, to further exploit information from the cross-section of funds. Furthermore, while we focus on the characteristics underlying the Fama-French factors because of the widespread use of the Fama-French models, our model can be readily extended to adjust performance for other characteristics such as those underlying the q-factors from \citet{hou2015digesting} or underlying the short- and long-run behavioral factors from \citet{daniel2017short}. 

subsection{old}
Growing concerns arise among academics and practitioners alike regarding the portfolio-based approach for identifying anomalies and, more generally, for testing asset-pricing models. This approach sorts stocks on a firm characteristic and constructs a zero-cost hedge portfolio which reflects the return differential between stocks in the top characteristic decile and stocks in the bottom characteristic decile. \citet{ang2008using} and \citet{fama2008dissecting}, among many others, argue that grouping stocks into portfolios and aggregating returns wastes and potentially distorts cross-sectional patterns in stock returns. Consequently, the Fama-French multi-factor models, that incorporate these portfolios as risk factors, do not fully adjust returns for the main anomalies. Instead of estimated factor betas many studies (e.g., \citet{avramov2006asset}, \citet{chordia2015cross}) have opted for individual firm characteristics to explain expected returns.
\par Our paper moves the debate of factor betas versus characteristics to the returns of mutual funds. We conduct monthly cross-sectional regressions
of fund returns on estimated factor betas and firm-level characteristics (aggregated to a fund-level statistic from a fund's holdings). Similar to the evidence on stock returns, we find that factor betas and characteristics each account for about half of the total model explained variation in fund returns. Thus, despite the mechanical relation one might expect between factor betas and their underlying characteristics, our results imply they do not convey identical information.  
\par An important application of the Fama-French factor models is the performance evaluation of mutual funds. Since these factor models do not fully account for the main anomalies, they only partially control for passive influences on fund returns. Our paper proposes a new performance measure that adjusts fund returns for both factor exposures and characteristics simultaneously. While it is debatable whether fund managers that actively shift their portfolios towards a certain characteristic dimension qualifies as skill, we believe that a performance measure should fully adjust for a particular anomaly. 
\par To calculate our double-adjusted performance measure, we propose a hierarchical Bayes structure in which we simultaneously model conditional factor model alphas and analyze the cross-sectional relation between fund alphas and characteristics. We find that characteristics are statistically significantly related to fund returns even after controlling for factor model betas, suggesting that performance identified via Fama-French factor models could be attributable to loadings on characteristics. When we reassess several previous studies on mutual fund relative performance, we find that our double-adjusted performance measure is a purer measure of skill when we analyze persistence in fund performance. Moreover, we find that relations between specific fund features (fund selectivity and fund flows) and fund performance is partially driven by the characteristic component of performance. All in all, by more fully adjusting performance for the influence of characteristics we might alter previous inference on relative mutual fund performance. 
\par Our hierarchical Bayes model can be extended in various directions. For example, the MIDAS approach of \citet{ghysels2005there} can be adopted to assign optimal weights to the daily returns in the window used to estimate the rolling window factor betas. In addition, one can impose a common structure on the factor betas too, rather than just the alphas, to further exploit information from the cross-section of funds. Furthermore, while we focus on the characteristics underlying the Fama-French factors because of the widespread use of the Fama-French models, our model can be readily extended to adjust performance for other characteristics such as those underlying the q-factors from \citet{hou2015digesting} or underlying the short- and long-run behavioral factors from \citet{daniel2017short}. 
\end{comment}